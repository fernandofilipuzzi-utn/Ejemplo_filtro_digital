\documentclass[informe.tex]{subfiles}
\begin{document}

Una forma de caracterizar un filtro digital en tiempo discreto es por medio de una ecuación de diferencias o por una función racional en la variable compleja $z$:
	\begin{equation}
		\label{eqn:filtro_digital:func_racional}	
		H(z) = \frac{
						\alpha \prod_{i=1}^{M}{(1-z_i z^{-1})}
					}{
						\prod_{k=1}^{N}{(1-p_k z^{-1})}
					}					
			= \frac{
						\prod_{i=0}^{M}{a_i z^{-i}}
					}{
						1 + \prod_{i=1}^{N}{b_k z^{-k}}
					}
	\end{equation}
donde  $N>M$ y
	\begin{tabbing}
		\phantom{$D_{n50}\ $}\= \kill
        \tab[0.25cm]$z_i$\> , son los ceros del numerador.\\
		\tab[0.25cm]$p_{k}$\> , son los ceros de polinomio del denominador. 
	\end{tabbing}	
	
Similar a los filtros analógicos, las funciones realizables tendrán la forma de Ec. \ref{eqn:filtro_digital:func_racional} y el problema que se presenta es el mismo que se encuentra en los filtros analógicos, el cual es que en este tipo de funciones no presentan un valor constante en ninguna banda.  Por ende se busca una función que aproxime algunos aspectos de las características deseadas.\\

Teniendo en cuenta la expresión $H(z)$ vista en la Ec. \ref{eqn:filtro_digital:func_racional} se pueden presentar dos enfoques de diseños de filtros.\\

- Filtros de Respuesta al Impulso infinita, IIR. Cuando $b_k \neq 0$ , la respuesta al impulso en tiempo discreto consiste de una serie infinita de términos. Para que el filtro sea estable requiere que todos sus polos estén dentro del circulo unitario en el plano z.\\

- Filtros de respuesta al impulso finita, FIR: en este caso, la Ec. \ref{eqn:filtro_digital:func_racional}, con $b_k=0$, tomará la forma:
	$$
		H(z)=\alpha \prod_{i=1}^{M} (1-z_i z^{-1})
	$$
Aquí, $M$ es una cantidad finita y de ahí surge su nombre de respuesta al impulso finito, ya que la respuesta al impulso $h(n)$ consiste de una serie finita de valores. También se desprende que sus características principales son la estabilidad y la causalidad, es decir: $	\sum_{n=-\infty}^{\infty}{ |h(n)|}<\infty$. 

\end{document}	