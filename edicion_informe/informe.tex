\documentclass[oneside,a4paper,10pt]{scrbook}
\usepackage[a4paper, includehead,
           inner=1cm, outer=1cm, top=2cm, bottom=2cm, 
           bindingoffset=1cm]{geometry}

\usepackage[utf8]{inputenc}
%\UseRawInputEncoding

\usepackage[spanish]{babel}

\usepackage{ragged2e} %

\usepackage{cancel} % cancelar terminos

\usepackage{blindtext}


%\usepackage[backend=biber]{biblatex}
%\usepackage{natbib}
%\usepackage[maxbibnames=99, sorting=none, backend=bibtex]{biblatex}
%\addbibresource{referencias.bib}
%\usepackage{sectionbib}
%\usepackage{chapterbib}
\usepackage[sectionbib]{natbib}
%\usepackage[nottoc,notlot,notlof]{tocbibind}
%\usepackage{etoolbox}
%\patchcmd{\thebibliography}{\chapter*}{\section}{}{}
%\usepackage[sectionbib]{chapterbib}

%\usepackage[pdftex]{graphicx}
\usepackage{graphicx}
\usepackage{grffile}

\usepackage{multicol}
\usepackage[spanish]{babel}
\usepackage{lipsum} 
\usepackage{subfiles}
\usepackage{multicol}
\usepackage{wrapfig}  
\usepackage{amsmath}
\usepackage[dvipsnames]{xcolor}
\usepackage{fancyvrb}
\usepackage{ulem}

\usepackage{amssymb} % el triangulito opensymbos
\usepackage{mathrsfs}
\usepackage{mathtools} %flecha doble para laplace

\usepackage{xcolor}
\usepackage{mdframed}

\usepackage{array,tabularx,calc}

\usepackage{hyperref}   %genera el hiperenlace en el menu

\usepackage{listings}
\usepackage[dvipsnames]{xcolor}

\usepackage{caption}
\usepackage{subcaption}

\usepackage{lastpage}
\usepackage{fancyhdr}
\usepackage{emptypage}


\newcommand\tab[1][1cm]{\hspace*{#1}}

\setcounter{secnumdepth}{5}



\graphicspath{ {./img/},{../design_matlab/output/},{./src_matlab/}  }

\usepackage{verbatim}

\DefineVerbatimEnvironment%
      {verbatimprog}%
      {Verbatim}%
      {fontsize=\footnotesize}%

\renewcommand{\footrulewidth}{0.3pt}

\definecolor{codegray}{rgb}{0.5,0.5,0.5}

% seteo del ligh
\lstdefinestyle{custommatlab}{
  backgroundcolor=\color{yellow!5},
  belowcaptionskip=1\baselineskip,
  breaklines=true,
  captionpos=b,                    
  keepspaces=true, 
  frame=L,
  xleftmargin=\parindent,
  language=Matlab,
  showstringspaces=false,
  showtabs=false,
  tabsize=2,
  numbers=left,        %numeración
  numbersep=5pt, 
  basicstyle=\footnotesize\ttfamily,
  keywordstyle=\bfseries\color{blue!40!black},
  commentstyle=\itshape\color{green!40!black},
  identifierstyle=\color{black},
  stringstyle=\color{purple},
  numberstyle=\tiny\color{codegray}
}

\lstset{escapechar=@\\,style=custommatlab}

\hypersetup{
	colorlinks,
	citecolor=black,
	filecolor=black,
	linkcolor=black,
	urlcolor=black
}


\begin{document}

	\pagestyle{fancy}
	
	\fancyhead{}\fancyfoot{}

	\fancyhead[LO,CE]{\textbf{Teoría de los circuitos II}}	
	
%	\fancyfoot[LO,LO]{\textbf{Filipuzzi, Fernando Rafael}}
%	\fancyfoot[RO,RO]{\thepage}
	\rfoot{ \thepage \hspace{1pt} - \pageref{LastPage}}
	\lfoot{ Filipuzzi, Fernando Rafael }
	
	\title{Teoría de los circuitos II - Ampliado}
	\subtitle{Trabajo final}
	\author{Filipuzzi, Fernando Rafael	}
	
	
	\maketitle
	
	\tableofcontents
	
	\listoffigures\newpage
   
%    \documentclass[informe.tex]{subfiles}
\begin{document}

\justify
El objetivo de este trabajo es llevar los aspectos teóricos de diseño y realización de filtros digitales a la práctica.\newline

En las primeras secciones se busca reunir las definiciones, expresiones y criterios que sirvan de referencia en el desarrollo de los script de diseño de los filtros en Matlab.\newline

En la secciones finales se trata la implementación de los filtros, describiéndose el hardware utilizado y las especificaciones de los filtros realizados.

\end{document}	
    
	\chapter{Introducción}
	{
		\subfile{intro.tex}
	}
	
	\clearpage	
	
	\chapter{Representación de las funciones del circuito}
	{
		\section{Comentarios sobre la bibliografía}
   		{
    		\subfile{representacion_bibliografia.tex}				
		}
	}	
	
    \clearpage	
	
	\chapter{Teoría de cuadripolos}
	{
		\section{Comentarios sobre la bibliografía}
   		{
    		\subfile{filtros_imagen_bibliografia.tex}				
		}
	}	
	
	\clearpage
	
		
	\chapter{Filtros por Teoría imagen}
	{
		\section{Comentarios sobre la bibliografía}
   		{
    		\subfile{filtros_imagen_bibliografia.tex}				
		}
	}	
	
	\clearpage
   
	\chapter{Filtros por funciones de aproximación}
	{
		\section{Introducción}
   		{
    		\subfile{filtros_funciones_intro.tex}			
		}
		
		\clearpage
		
    	\section{Filtro de Butteworth}
   		{
    		\subfile{filtros_funciones_butterworth.tex}					
		}
		
		\clearpage
		
		\section[]{Filtro de Chebyshev}
   		{
    		\subfile{filtros_funciones_chebyshev}	
		}
		
		\clearpage
		
		\section{Filtro de Bessel}
   		{
    		\subfile{filtros_funciones_bessel.tex}		
		}
		
		\clearpage
		
		\section{Comentarios sobre la bibliografía}
   		{
    		\subfile{filtros_funciones_bibliografia.tex}				
		}
		
		\clearpage		
		
			
		%\addcontentsline{toc}{section}{Comentarios sobre la bibliografía}
%			%\bibliographystyle {plain}
		%	\bibliographystyle{apalike}
%			\bibliography{referencias}
%			\nocite{everitt1961ingenieria}
	}
	
	\clearpage
	
    \chapter{Transformación de frecuencia}
	{
		\section{Introducción}
   		{
    		\subfile{transformacion_frecuencia_intro.tex}		
		}
		
		\clearpage	
		
		\section{Transformación de paso bajo a paso alto}
   		{
    		\subfile{transformacion_frecuencia_hp.tex}		
		}
		
		\clearpage
		
		\section{Transformación de paso bajo a pasa banda}
   		{
    		\subfile{transformacion_frecuencia_bp.tex}		
		}
		
		\clearpage
		
		\section{Comentarios sobre la bibliografía}
   		{
    		\subfile{transformacion_frecuencia_bibliografia.tex}				
		}		
		
	}
	
	\clearpage
	
    \chapter{Realización de redes sin perdidas}{
    
    	\section{Funciones de impedancia y admitancia}	
 	   {
    	}\newpage
    
		\section{Función de transferencia}	
		{
	   }\newpage    
   
    }
    
    \clearpage
        
	\chapter{Diseño de Filtros digitales}{
	
		\section{Introducción}
   		{
    		\subfile{filtros_digitales_intro.tex}				
		}
		
		\clearpage
		
		\section{Filtros IIR}
   		{
    		\subfile{filtros_digitales_iir.tex}					
		}
		
		\clearpage
		
		\section{Filtros FIR}
   		{
    		\subfile{filtros_digitales_fir.tex}					
		}
		
		\clearpage
		
		\section{Comentarios sobre la bibliografía}
	   	{
    		\subfile{filtros_digitales_bibliografia.tex}					
		}
	}
	
	\clearpage
	
	\chapter{Realización de filtros Digitales}
	{
		\section{Realización}
		{
			\subfile{realizacion_filtros_digitales.tex}
		}
		
		\clearpage	
		
		\section{Comentarios sobre la bibliografía}
		{
			\subfile{realizacion_filtros_digitales_bibliografia.tex}
		}	
	}
	
	\clearpage
%%	
	\chapter{Procesamiento digital-construcción de filtros digitales}
	{	
		\section{Realización}
		{
			\subfile{construccion_filtros_digitales.tex}
		}
		
		\clearpage
		
		\section{Comentarios sobre la bibliografía}
		{
			\subfile{construccion_filtros_digitales_bibliografia.tex}
		}			
	}
	
	\clearpage
%%	
	\chapter{Especificaciones de las implementaciones de los filtros realizados}
	{
		\section{Especificaciones generales del hardware utilizado}
		{	
			\subfile{disenio_y_construccion_hardware.tex}
		}
		
		\clearpage
		
		\section{Especificaciones de los filtros IIR}
		{
		
			\subsubsection{Filtro de Butterworth  (microcontrolador)}
			{
					\subfile{disenio_y_construccion_intro.tex}
			}
		
			\clearpage		
		
			\subsection{Filtros pasa bajo}
			{
				\subsubsection{Filtro de Butterworth  (microcontrolador)}
				{
					\subfile{disenio_y_construccion_iir_butterworth_lp_up.tex}
				}
				
				\clearpage	
							
				\subsubsection{Filtro de Chebyshev (microcontrolador)}
				{
					\subfile{disenio_y_construccion_iir_chebyshev_lp_up.tex}	
				}
								
				\clearpage	
				
				\subsubsection{Filtro de Butterworth (FPGA)}
				{
					\subfile{disenio_y_construccion_iir_butterworth_lp_fpga.tex}	
				}
			}
			
 			\clearpage	
			
			\subsection{Filtros pasa banda}
			{
				\subsubsection{Filtro de Butterworth  (microcontrolador)}
				{
					\subfile{disenio_y_construccion_iir_butterworth_bp_up.tex}
				}
			}
	
			\clearpage				
			
			\subsection{Filtros pasa alto}
			{
				\subsubsection{Filtro de Butterworth  (microcontrolador)}
				{
					\subfile{disenio_y_construccion_iir_butterworth_hp_up.tex}			
				}
			}					
		}
		
		\clearpage			
		
		\section{Especificaciones de los filtros FIR}
		{
			\subsection{Filtro pasa bajo}
			{
				\subfile{disenio_y_construccion_fir_lp.tex}
			}
			
			\clearpage			

			\subsection{Filtro pasa alto}
			{
				\subfile{disenio_y_construccion_fir_hp.tex}
			}
		}
	}
	
	\clearpage			
	
	\chapter{Filtro Activo}
	{
		
		\section{Comentarios sobre la bibliografía}
		{
			\subfile{filtro_activo}
		}
		
		\clearpage			
	}	

	\clearpage			
	
%	\documentclass[informe.tex]{subfiles}
\begin{document}

W. L. Everitt. G. E. Anner. Ingenieria de comunicaciones\newline
sobre teoría imagen

\end{document}	 
%		\chapter{Bibliografía adicional}
%		{
%			\subfile{bibliografia_adicional.tex}
		
%		
%		\clearpage

	
%	\bibliographystyle{apalike}
%	\bibliography{referencias}
%	\nocite{everitt1961ingenieria}
%	\phantomsection
%	\addcontentsline{toc}{section}{Comentarios sobre la bibliografía}			
%		}
\end{document}