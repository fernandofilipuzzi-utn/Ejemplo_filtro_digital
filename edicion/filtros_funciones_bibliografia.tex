\documentclass[informe.tex]{subfiles}
\begin{document}

Lam Harry. Analog and digital filters design and realization.\newline 
\tab[1cm]\textit{ Desarrollo general.}\newline

Pactitis S. A. Active Filters: Theory and design\newline
\tab[1cm]\textit{ Desarrollo del Filtro de Chebyshev.}\newline

Weinberg. Network analysis and synthesis\newline
\tab[1cm]\textit{En la pág. 365, tiene mejor desarrollado la introducción de funciones de aproximación.\newline
\tab[1cm]Las expresiones para calcular los polos del filtro de Chebyshev. }\newline

Kendal Su. Analog Filters.\newline
\tab[1cm]\textit{ En la pág.66, el desarrollo por igualación de parámetros de Bessel.}\newline
\tab[1cm]\textit{ En la pág.69, la expresión cerrada de Bessel.}\\

Van Valkenburg, Mac Elwyn. Analog filter design.\newline
\tab[1cm]\textit{En la pág 294. Función de transferencia y retardo en el tiempo}\newline

Arthur Williams, Fred J. Taylor.  Electronic Filter Design Handbook. Fourth Edition.\newline
\tab[1cm]\textit{ En la página 52, menciona sobre la ubicación geométrica de los polos.}\newline

Van Valkenburg, Mac Elwyn. Introduction to modern network synthesis.\newline
\tab[1cm]\textit{En la pág 391, Fig. 13-17 Comparación de la ubicación geométrica de los polos entre filtros de máxima planicidad de retardo y de magnitud}\newline

Daryanani G. Principles of active network synthesis and design.\newline
\tab[1cm]\textit{Ejemplo de filtros de Bessel}\newline



\end{document}	